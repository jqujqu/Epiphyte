\documentclass[11pt]{article}

\usepackage{fullpage,times,pgf,enumerate,float,amsmath}

\title{Maximum likelihood Phylogenetic model}
\author{Jenny Qu}

\begin{document}

\maketitle
\section{Model and notation}

A methylome is a sequence of CpG sites (or bins of CpG sites) with
each site displaying a methylation state from the set
$\{0,1\}$. During evolution, two processes jointly govern the methylation
state inheritance across generations, and the correlation between neighboring
sites within each generation.

The inheritance process is described with a continuous-time Markov
process over the state space $\{0,1\}$. Let the initial distribution
be $\pi=(\pi_0, \pi_i)$. Let the transition rate matrix matrix be
$$Q=\begin{bmatrix}
      -\lambda  & \lambda  \\[0.3em]
      \eta & -\eta
    \end{bmatrix}.
$$
In continuous-time Markov process, the transition probability between
two time points separated by time $t$ is determined by two terms
$(t(\lambda+\eta), \lambda/\eta)$. Therefore, we let $\eta =
1-\lambda$, $\lambda\in(0,1)$. The transition probability matrix becomes
$$P(t) = exp(Qt) = \begin{bmatrix}
      1-\lambda T & \lambda T \\[0.3em]
      (1-\lambda)T & 1-(1-\lambda)T
    \end{bmatrix},$$
where $T = 1-exp(-t) \in(0,1)$ for $t>0$.

The autocorrelation within a methylome is described with a
discrete-time (corresponding to discrete CpG sites) Markov chain over
the state space $\{0,1\}$. The initial distribution is $\pi$ (only
relevant in the root species). The transition probability matrix,
$$G=\begin{bmatrix}
      g_0  & 1-g_0  \\[0.3em]
      1-g_1 & g_1
    \end{bmatrix}
$$
is assumed to be homogeneous in all species (this assumption can be
relaxed).

For a given set of extant species and their phylogenetic relationship,
we use $\tau=\{V, E\}$ to denote the phylogenetic tree, including the
set of nodes, and the set of edges $E$. We assume that the tree
structure is given, and the branch lengths are unknown parameters.

The model parameter space is thus $$\Theta=\{E,\lambda, g_0, g_1,
\pi_0\},$$ where $E$ is the collection of branch lengths, and
$\lambda$ is the transition rate from $0$ to $1$ in the
continuous-time Markov process. $\pi_0$ is the hypomethylation
probability in the root node.


The two Markov processes are combined in the following way. The
direction of inheritance process is obvious. For the discrete Markov
process, let the first CpG site from the 5' end of the '+' strand to
be the start position of a chain. The root methylome has no ancestor,
so it is modeled only with the discrete time Markov chain described by
$\{\pi, G\}$. For CpG sites at start position of a DNA fragment, the
evolution of methylation states on this site is modeled only with the
inheritance process described with $\{\pi, Q, \tau\}$. For CpG sites
from internal nodes and leaf nodes that are not the start of a DNA
fragment, we will combine the two processes to model its evolution. Let
one such site be position $j$ in node $v$. Let $i=j-1$ be the previous
site in the DNA fragment. Let node $u$ be $v$'s parent in the
phylogenetic tree. Use the notation $u_j$ to denote the methylation state
of node $u$ at position $j$, $u_j\in\{0,1\}$. Let $t_v=e(u,v)$ be the
branch length between nodes $u$ and $v$. We define the following
transition probability:
\begin{equation}
\Pr(v_j|v_i,u_j)=  \frac{G_{v_iv_j}P(t_v)_{u_jv_j}}{\sum_{s=0,1}G_{v_is}P(t_v)_{u_js}}
\end{equation}

Assume hypomethylation probabilities at $N$ sites are observed in the
methylome at each leaf node of a phylogenetic tree. Let $O_i$ be the
observed hypomethylation probabilities at location $i$ in all extant
species, $i=1,\ldots, N$. Let $O=O_1O_2\cdots O_N$ be the total
observed methylation states at leaf nodes.  Let $S=S_1\cdots S_N$ be
the unobserved methylation states at internal nodes.

\section{Incomplete data}
We have observations of bisulfite read counts, or precomputed
hypomethylation probability at leaf nodes. There is no prior
information about the methylation states at internal nodes.

\subsection{Data Likelihood}
The above description defines a hidden Markov model over the
space $$H=\{h\in\{0,1\}^{|V|}: \text{all possible combination of
methylation states over the phylogenetic tree} \}.$$ Each state
specifies the history of the methylation evolution at CpG site, and
thus we call it History of Methylation Evolution (HME). The transition
probability between two HMEs can be calculated using transition
probabilities in the previous section. Let $h_i$, $h_j$ be the HMEs at
two neighboring sites in the genome. Let the root node be $r$. Let $v$
be an internal or leaf node in the phylogenetic tree, and let its
parent node be $u$. Let $r_i$ be the methylation state at node $r$ as
specified by HME $h_i$. Let the HME transition probability matrix be
$A=\{a_{ij}\}$, where
$$a_{ij}=\Pr(h_j|h_i) = G_{r_ir_j}\prod_{(u,v)\in E}P(v_j|v_i,u_j).$$
The data likelihood calculation can be written as
\begin{equation}
L = P(O|\Theta) =\sum_{H_1,\ldots, H_N\in H} \Pr(H_1)\Pr(O_1|H_1)\prod_{i=1}^{N} \Pr(H_i| H_{i-1}, \Theta)\Pr(O_i|H_i),
\end{equation}
where $H_t$ is a HME random variable at position $t$. $H$ is the HME
space.

\subsection*{Forward algorithm}
Alternatively, we can use the forward algorithm for likelihood
calculation. Define the forward variables as below:
\begin{equation}
\begin{aligned}
\alpha_1(i) &= \Pr(O_1, H_1=h_i|\Theta) = \Pr(h_i|\Theta)\Pr(O_1|h_i)\\
\alpha_{n+1}(j) & = \sum_{i=1,\ldots, |H|} \alpha_n(i)a_{ij} \cdot \Pr(O_{n+1}|h_i)
\end{aligned}
\end{equation}
The data likelihood is thus
$$L = P(O|\Theta) = \sum_{i=1,\ldots, |H|}\alpha_{N}(i).$$
And the log-likelihood is
$$l = \log{L} = \log\sum_{i=1,\ldots, |H|}\alpha_{N}(i).$$

\subsection{First order derivative}
Our parameter space is $\Theta=\{E,\lambda, g_0, g_1, \pi_0\}$.
Let $E'= \{e'= 1-exp(-e):\forall e\in E\}$. There is a 1-1 mapping
between $E$ and $E'$, so the model parameter space is equivalently
$\Theta=\{E',\lambda, g_0, g_1, \pi_0\}$. For simplicity, we use the
same notation $\Theta$. Let $x$ be a parameter of interest.

$$\frac{\partial l}{\partial x}
= \frac{1}{L}\sum_{i=1,\ldots, |H|}\frac{\partial }{\partial x} \alpha_{N}(i) $$


\noindent\textbf{Recursion by forward algorithm}
Using the recursive definition of forward variables,
$$\frac{\partial }{\partial x} \alpha_{n+1}(j) =
\sum_{i=1,\ldots, |H|} \big[\frac{\partial }{\partial x}\alpha_n(i) \cdot a_{ij} +\alpha_n(i)\cdot \frac{\partial }{\partial x}a_{ij} \big] \cdot \Pr(O_{n+1}|h_j), $$
the first order derivative computation comes down to computing
derivatives for HME transition probabilities $a_{ij}$, and the base case
forward variables $\alpha_1(i)$, where $i, j \in \{1, \ldots, |H|\}$.
\\
\\
\noindent\textbf{Some Preparation}
The continuous-time Markov process' transition probability matrix
corresponding to a given branch $(u,v)$ contains two parameters:
$\lambda$, and $T_u=1-exp(-t_u)$. The derivatives with respect to
these two parameters are
\begin{equation}
\begin{aligned}
\frac{\partial}{\partial \lambda} P(t)
&= T\cdot\begin{bmatrix}
  -1 & 1 \\[0.3em]
  -1 & 1
\end{bmatrix}\\
\frac{\partial}{\partial T} P(t) &= Q
\end{aligned}
\end{equation}
where $T = 1-exp(-t)$.


The discrete-time Markov chain involves two parameters, $g_0$ and $g_1$.
\begin{equation}
\begin{aligned}
\frac{\partial}{\partial g_0} G_{ij} &= \delta_{\{i=0\}}\cdot (2*\delta_{\{j=0\}}-1)\\
\frac{\partial}{\partial g_1} G_{ij} &= \delta_{\{i=1\}}\cdot (2*\delta_{\{j=1\}}-1)
\end{aligned}
\end{equation}

The combined transition probability $\Pr(v_j|v_i,u_j)$  contains parameter $T_v, g_{v_i}$, and $\lambda$.
\begin{equation}
\begin{aligned}
\frac{\partial}{\partial T_v}\Pr(v_j|v_i,u_j)
&=\frac{ G_{v_iv_j}\cdot\frac{\partial}{\partial T_v}P(t_v)_{u_jv_j}} {[\sum\limits_{s=0,1}G_{v_is}P(t_v)_{u_js}]} -
 \frac{G_{v_iv_j}P(t_v)_{u_jv_j}\cdot [\sum\limits_{s=0,1}G_{v_is}\frac{\partial}{\partial T_v}P(t_v)_{u_js}] } {[\sum\limits_{s=0,1}G_{v_is}P(t_v)_{u_js}]^2} \\
\frac{\partial}{\partial g_{v_i}}\Pr(v_j|v_i,u_j)
&=\frac{ \frac{\partial}{\partial g_{v_i}}G_{v_iv_j}\cdot P(t_v)_{u_jv_j}}{[\sum\limits_{s=0,1}G_{v_is}P(t_v)_{u_js}]} -
 \frac{G_{v_iv_j}P(t_v)_{u_jv_j}\cdot [\sum\limits_{s=0,1}\frac{\partial}{\partial g_{v_i}}G_{v_is}P(t_v)_{u_js}] } {[\sum\limits_{s=0,1}G_{v_is}P(t_v)_{u_js}]^2} \\
\frac{\partial}{\partial \lambda}\Pr(v_j|v_i,u_j)
&=\frac{ G_{v_iv_j}\cdot\frac{\partial}{\partial \lambda}P(t_v)_{u_jv_j}} {[\sum\limits_{s=0,1}G_{v_is}P(t_v)_{u_js}]} -
 \frac{G_{v_iv_j}P(t_v)_{u_jv_j}\cdot [\sum\limits_{s=0,1}G_{v_is}\frac{\partial}{\partial \lambda}P(t_v)_{u_js}] } {[\sum\limits_{s=0,1}G_{v_is}P(t_v)_{u_js}]^2} \\
&=\frac{ G_{v_iv_j}T_v\cdot(2\delta_{\{v_j=1\}}-1)} {[\sum\limits_{s=0,1}G_{v_is}P(t_v)_{u_js}]} -
 \frac{G_{v_iv_j}P(t_v)_{u_jv_j}T_v [\sum\limits_{s=0,1}G_{v_is}(2\delta_{\{s=1\}}-1)] } {[\sum\limits_{s=0,1}G_{v_is}P(t_v)_{u_js}]^2}
\end{aligned}
\end{equation}


\noindent\textbf{Derivative of $\lambda$}
\begin{equation*}
\begin{aligned}
\frac{\partial}{\partial\lambda}\alpha_1(i)
&= \alpha_1(i)\times\big[\sum_{(u,v)\in E} \frac{T_v\cdot(2\delta_{\{v_i=1\}}-1)}{P(t_v)_{u_iv_i}} \big] \\
\frac{\partial}{\partial\lambda}a_{ij}
&= a_{ij}\times\big[\sum_{(u,v)\in E}\frac{\frac{\partial}{\partial\lambda}P(v_j|v_i,u_j)}{P(v_j|v_i,u_j)} \big]
\end{aligned}
\end{equation*}

\noindent
\textbf{Derivative of branch length $T_v$}
\begin{equation*}
\begin{aligned}
\frac{\partial}{\partial T_v}\alpha_1(i) &= \alpha_1(i)\times\frac{Q_{u_iv_i}}{P(t_v)_{u_iv_i}} \\
\frac{\partial}{\partial T_v}a_{ij} &=  a_{ij}\times\frac{\frac{\partial}{\partial T_v}\Pr(v_j|v_i,u_j) }{\Pr(v_j|v_i,u_j)}
\end{aligned}
\end{equation*}

\noindent\textbf{Derivative of transition probability $g_s$}
\begin{equation*}
\frac{\partial}{\partial g_s}a_{ij} =
a_{ij} \times\big[ \delta_{\{r_i=s\}}\times\frac{(2\delta_{\{v_j=s\}}-1)}{G_{r_ir_j}} + \sum_{(u,v)\in E} \delta_{\{s=v_i\}}\times\frac{\frac{\partial}{\partial g_{v_i}}P(v_j|v_i,u_j)}{P(v_j|v_i,u_j)} \big]
\end{equation*}

\noindent\textbf{Derivative of initial distribution $\pi_0$}
\begin{equation*}
\frac{\partial}{\partial \pi_0}\alpha_1(i) = \alpha_1(i)\times\frac{2\delta_{\{r_i=0\}}-1}{\pi_{r_i}}
\end{equation*}

All other partial derivatives of $\alpha_n(i)$ and $a_{ij}$ that are
not specified above have value 0.

\subsection{Reduce transition complexity}
The total number of possible transitions between HMEs is $2^{2|V|}$. We
introduce constraints on transition between HMEs to reduce the
computational complexity.

\textbf{Neighbor HME} Let $N(h)=\{h'\in H: P(H_{n+1}=h'|H_n=h) > 0\}$,
be the neighbor set of an HME $h$. We always allow self-transition,
\textit{i.e.} $h\in N(h), \forall h\in H$. For any two HMEs, they are
neighbors if one HME can be converted to the other HME by picking a
subtree and forcing all nodes in the subtree to have a same
methylation state.

By this definition, the complete hypomethylated or hypermethylated HME
is neighbor with all HMEs. Different HMEs might have different number
of neighbors.  The neighbor definition implies a symmetric
relationship, \textit{i.e.} if $h\in N(h')$ then $h'\in N(h)$. The
restricted transition probabilities are defined as below:
$$a'_{ij} =1_{h_j\in N(h_i)} \frac{a_{ij}}{\sum_{k\in N(i)} a_{ik}}$$.
The relevant partial derivatives should be recalculated as:
\begin{gather*}
\frac{\partial}{\partial x} a'_{ij} = 1_{h_j\in N(h_i)}
\bigg[ \frac{\frac{\partial}{\partial x} a_{ij}}{\sum_{k\in N(i)} a_{ik}} -
 \frac{\sum_{k\in N(i)}\frac{\partial}{\partial x} a_{ik}}{\big(\sum_{k\in N(i)} a_{ik}\big)^2 }\bigg]
\end{gather*}
where parameter $x\in\{\lambda, g_0, g_1\}\bigcup\{T_v:v\in V\}$.



\section{Approximation method}
In the incomplete data situation, all possible transitions between HMEs
have to be considered, which give rise to a complexity of $O(c^n)$ where
$n$ is the number of nodes in the phylogenetic tree.

We would like to get around the computational complexity by
approximating the transition probabilities between HME with a product
of weighted transition probabilities along each branch, with the
weight being a function of the posterior hypomethylation probabilities
at relevant nodes. The work flow is described below:
\begin{enumerate}
\item Initialize hypomethylation probability at internal nodes
\item Compute likelihood and gradient
\item Optimize parameter
\item Update posterior hypomethylation probability at internal nodes using
Markov blanket and the updated model parameters
\item Repeat step 2-4 until convergence of parameter.
\end{enumerate}


\subsection{Likelihood and gradient}
Let $s^{(i)}_u\in\{0,1\}$ be the methylation sate of node $u$ on the
phylogenetic tree at position $i$ in the methylome.  Let
$p^{(i)}_{s_u}$ be the observed or posterior probability of node $u$
having methylation state $s_u$ at position $i$.

Likelihood and gradient for the first site in the genomic fragment is
\begin{gather*}
L_0=\sum_{s_r} p_{s_r}\pi_{s_r} \prod_{(u,v)\in E}\big[\sum_{s_u, s_v} p_{s_u}p_{s_v}\Pr(T_{u,v})_{s_us_v} \big]\\
\frac{\partial \log L_0}{\partial x} = \frac{1}{L_0} \sum_{s_r} \bigg(p_{s_r}\pi_{s_r} \prod_{(u,v)\in E}\big[\sum_{s_u, s_v} p_{s_u}p_{s_v}\Pr(T_{u,v})_{s_us_v} \big]\bigg)\times
\bigg( \frac{\frac{\partial \pi_{s_r}}{\partial x}}{\pi_{s_r}} +  \sum_{(u,v)\in E}\frac{\sum_{s_u, s_v} p_{s_u}p_{s_v}\frac{\partial\Pr(T_{u,v})_{s_us_v}}{\partial x} }{\sum_{s_u, s_v} p_{s_u}p_{s_v}\Pr(T_{u,v})_{s_us_v} } \bigg)
\end{gather*}
Transition probability between site $i$ and $i+1$ is
\begin{gather*}
L_{i,i+1} = \sum_{s^{(i-1)}_r s^{(i)}_r}p_{s^{(i)}_r}p_{s^{(i+1)}_r} G_{s^{(i)}_rs^{(i+1)}_r}  \prod_{(u,v)\in E}
\big[\sum_{s^{(i)}_vs^{(i+1)}_us^{(i+1)}_v }  p^{(i)}_vp^{(i+1)}_up^{(i+1)}_v \Pr(s^{(i+1)}_v | s^{(i)}_v, s^{(i+1)}_u ) \big] \\
\frac{\partial}{\partial x} \log L_{i,i+1} = \frac{1}{L_{i,i+1}}
\sum_{s^{(i)}_r s^{(i+1)}_r} \bigg( p_{s^{(i)}_r}p_{s^{(i+1)}_r} G_{s^{(i)}_rs^{(i+1)}_r}  \prod_{(u,v)\in E}
\big[\sum_{s^{(i)}_vs^{(i+1)}_us^{(i+1)}_v }  p^{(i)}_vp^{(i+1)}_up^{(i+1)}_v \Pr(s^{(i+1)}_v | s^{(i)}_v, s^{(i+1)}_u ) \big]  \bigg) \times \\
\bigg( \frac{\frac{\partial}{\partial x} G_{s^{(i)}_rs^{(i+1)}_r}}{G_{s^{(i)}_rs^{(i+1)}_r}}  +
      \sum_{(u,v)\in E}\frac{\sum_{s^{(i)}_vs^{(i+1)}_us^{(i+1)}_v }  p^{(i)}_vp^{(i+1)}_up^{(i+1)}_v \frac{\partial}{\partial x}\Pr(s^{(i+1)}_v | s^{(i)}_v, s^{(i+1)}_u )}{\sum_{s^{(i)}_vs^{(i+1)}_us^{(i+1)}_v }  p^{(i)}_vp^{(i+1)}_up^{(i+1)}_v \Pr(s^{(i+1)}_v | s^{(i)}_v, s^{(i+1)}_u )}   \bigg)
\end{gather*}
The overall log-likelihood  and gradients are
\begin{gather*}
\log L = \log L_0 + \sum_{i=0}^{N-1} \log L_{i,i+1} \\
\frac{\partial}{\partial x} \log L = \frac{\partial}{\partial x}\log L_0 + \sum_{i=0}^{N-1}\frac{\partial}{\partial x}\log L_{i,i+1}
\end{gather*}














\bibliographystyle{plain}
\bibliography{biblio}

\end{document}
