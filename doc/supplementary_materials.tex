\documentclass[11pt]{article}

\usepackage{fullpage,times,pgf,amsmath}
\usepackage{authblk}

\newcommand{\child}[1]{\ensuremath{\mathrm{child}(#1)}}
\newcommand{\tsum}{{\textstyle\sum}}
\newcommand{\tprod}{{\textstyle\prod}}


\title{Supplementary Material: Evolution of mammalian sperm methylomes}
\date{}
\author{Jenny Qu \and Egor Dolzhenko \and Other People \and Andrew D Smith}
%\affil[1]{Molecular and Computational Biology Section, Division
 %         of Biological Sciences, University of Southern California,
  %        Los Angeles, CA, 90089, United States of America}

\begin{document}

%% \maketitle

%% \section*{Definitions for enhancers}



%% \section*{Defining mutually orthologous regions for a set of species}



%% \section*{Identifying HMRs that appear to be constant across cell types in human and mouse}

%% We require a set of HMRs that we can assume have constant size between
%% cell types, when they exist. We have this information for mouse and
%% for human from multiple somatic cell types. For chimp we also have a
%% few more.

%% \section*{Mapping sites between pairs of species}







%%%%%%%%%%%%%%%%%%%%%%%%%%%%%%%%%%%%%%%%%%%%%%%%%%%%%%%%%%%%%%%%%%%%%%%%
%%%%%%%%%%%%%%%%%%%%%%%%%%%%%%%%%%%%%%%%%%%%%%%%%%%%%%%%%%%%%%%%%%%%%%%%
%%%%%%%%%%%%%%%%%%%%%%%%%%%%%%%%%%%%%%%%%%%%%%%%%%%%%%%%%%%%%%%%%%%%%%%%
%%%%%%%%%%%%%%%%%%%%%%%%%%%%%%%%%%%%%%%%%%%%%%%%%%%%%%%%%%%%%%%%%%%%%%%%
%%%%%%%%%%%%%%%%%%%%%%%%%%%%%%%%%%%%%%%%%%%%%%%%%%%%%%%%%%%%%%%%%%%%%%%%


\section{A phylo-epigenomic model with independent sites}

A methylome is a sequence of CpG sites with each site displaying a
methylation state from the set $\{0,1\}$.
%%% ADS: not sure if the statement below can be considered accurate or
%%% an assumption.

%%% the Markov process
Our goal here is to model the evolution of DNA methylation states
between species. We first focus on a single-site, and then extend the
model to multiple sites but for which epigenomic evolution is assumed
to be independent. An individual CpG site has a methylation state that
follows a continuous-time Markov process over the state space $\{0,
1\}$, representing hypermethylated and hypomethylated states. Let
$\pi=(\pi_0, \pi_1)$ be the initial distribution for the process, and
let the transition rate matrix matrix be
\[
Q=\begin{bmatrix}
-\lambda & \lambda\\
    \eta & -\eta
\end{bmatrix}.
\]
In a continuous-time Markov process the transition probability between
two time points separated by time $t$ is determined by two terms
$(t(\lambda + \eta), \lambda/\eta)$. The transition probability matrix
is
\[
P(t) = \exp(Qt) =
\begin{bmatrix}
  1 - \lambda T & \lambda T \\
         \eta T & 1 - \eta T
\end{bmatrix},
\]
where $T = 1 - \exp(-t)$, so $T\in (0,1)$ for $t > 0$.

%%% the tree and our notation
For a given set of extant species and with an assumed set of
phylogenetic relationships, {\it i.e.} a tree topology, we use $\{V,
E\}$ to denote the sets of vertices and of edges in the tree. The
model parameter space is thus
\[
\Theta=\{E,\lambda\},
\]
where $E$ represents the lengths of the edges in the phylogenetic
tree, and $\lambda$ is the transition from hypomethylated state to
hypermethylated state. Below we will use $u$, $v$ and $w$ to denote
tree nodes, with $(u, v)\in E$ and $(v, w)\in E$ implicit. Let $s(u)
\in \{0, 1\}$ denote the state of node $u$, and let $R_u$ be the set
of methylation states at leaf nodes that are descendents of $u$.  We
will use letters $x$ and $y$ to denote methylation states, and $x$
will typically denote the state at the parent of the node whose state
is represented by $y$. Finally, we use $\ell_{uv}$ to denote the
length of edge $(u, v)\in E$.

%% The likelihood is then
%% \begin{equation}
%% L = \Pr(X|\Theta) = \prod_{i=1}^{N} P(X_i|\Theta)
%% \end{equation}

%% Let $u$ be the probability that node $u$ has hyper-methylated state.
%% If $u$ is a leaf node, then the probability is known beforehand; when
%% $u$ is an internal node this probability is not generally known.

%%% the tree and our notation
For node $v$ with parent $u$ having methylation state $s(u) = x$,
\[
p_{x}(v) = \Pr(R_v | s(u) = x, \Theta),
\]
is the probability of observing states $R_v$ at terminal descendents
of node $v$. For notational convenience we define
\[
q_y(v) = \left\{
\begin{array}{ll}
  \Pr(s(v) = y) & \mbox{if $v$ is a leaf node,} \\
  \prod_{w\in \child{v}}  p_{y}(w) & \mbox{otherwise.} \\
\end{array}\right.
\]
We can then write the probability $p_x(v)$ as the recurrence
\[
p_{x}(v) = \sum_{y}\Big(P(\ell_{uv})_{xy} \times q_y(v)\Big),
\]
where $\ell_{uv}$ is the branch length between $u$ and $v$. With node
 $r$ representing the root of the tree, we can express the likelihood
for the entire tree as
%%
\begin{equation}
L = L(\Theta|R_r) = \Pr(R_r|\Theta) = \sum_{x}\pi_{x}q_x(r).
\end{equation}
This recurrence is the basis of Felsenstein's pruning algorithm for
efficiently computing the likelihood of a tree topology, branch
lengths and transition rate, given data at leaf nodes.

\section{Derivatives required for parameter optimization}

The derivative of the transition rate matrix $P(t)$ with respect to
rate $\lambda$ is
\[
P_\lambda (t) = \frac{\partial}{\partial \lambda} P(t)
= T\begin{bmatrix}
  -1 & 1 \\
  -1 & 1
\end{bmatrix},
\]
where as above $T = 1-\exp(-t)$. The derivative of the likelhood with
respect to $\lambda$ satisfies the recurrence relation
\begin{equation}
\frac{\partial p_x(v)}{\partial \lambda}
= \sum_{y} \bigg(P_{\lambda}(\ell_{uv})_{xy} q_y(v) +
P(\ell_{uv})_{xy} \sum_{w\in \child{v}} \frac{\partial p_{y}(w)}{\partial \lambda}\frac{q_y(v)}{p_{y}(w)}\bigg).
\end{equation}
Incorporating the probabilities associated with the root node $r$, the
derivative of the likelihood over the entire tree becomes
\begin{equation}
  \frac{\partial}{\partial \lambda}L = \sum_{x} \pi_{x}\frac{\partial q_x(r)}{\partial \lambda}.
\end{equation}

We will also require derivatives of the likelihood with respect to the
lengths of each tree branch. The derivative of the rate matrix with
respect to branch length can be computed
\[
P_T (t) = \frac{\partial}{\partial T} P(t) = Q.
\]
Let $T_{uv} = 1 - \exp(-\ell_{uv})$. Then the derivative of $p_x(v)$
for specific methylation state $x$ is
\begin{equation}
\frac{\partial p_x(v)}{\partial T_{uv}} =
\sum_{y}
\Big(P_{T_{xy}}(\ell_{uv})_{xy} q_y(v) +
     P(\ell_{uv})_{xy}\sum_{w\in \child{v}}
     \frac{\partial p_y(w)}{\partial T_{uv}}\frac{q_y(v)}{p_y(w)}\Big).
\end{equation}
When we incorporate the probabilities associated with the root node
the derivative of the likelihood over the entire tree becomes
\begin{equation}
\frac{\partial}{\partial T_{uv}} L
=\sum_{x} \pi_{x} \frac{\partial q_x(r)}{\partial T_{uv}}.
\end{equation}
Finally, the derivative of the likelihood with respect to the
methylation level $\pi_0 = 1 - \pi_1$ at root node $r$ is
\[
\frac{\partial}{\partial \pi_0} = q_{0}(r) - q_1(r).
\]

\section{Considering multiple sites}

Let $R^{(i)}$ denote the set of methylation states associated with
terminal nodes at CpG site $i$, with $1\leq i\leq N$. Assuming the
epigenomic states at distinct sites evolve independently according to
a single continuous time Markov process, define
\[
L_i = L(\Theta|R^{(i)}_r).
\]
Then the likelihood for the entire sequence of sites is
\[
L(\Theta|R_r) = \prod_{i=1}^NL(\Theta|R^{(i)}_r).
\]
The relevant partial derivatives of this likelihood all have a common form:
\begin{gather*}
\frac{\partial L}{\partial \lambda} =
\Big(\tprod_{i=1}^N L_i\Big)\Big(\tsum_{i=1}^N \frac{\partial L_i/\partial \lambda}{L_i}\Big),\quad
\frac{\partial L}{\partial T_{uv}} =
\Big(\tprod_{i=1}^N L_i\Big)\Big(\tsum_{i=1}^N \frac{\partial L_i/\partial T_{uv}}{L_i}\Big)\quad \mbox{and}\\
\frac{\partial L}{\partial \pi_0} =
\Big(\tprod_{i=1}^N L_i\Big)\Big(\tsum_{i=1}^N \frac{\partial L_i/\partial \pi_0}{L_i}\Big).
\end{gather*}

\bibliographystyle{plain}%namedplus}
\bibliography{biblio}

\end{document}

