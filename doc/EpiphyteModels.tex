\documentclass[11pt]{article}

\usepackage{fullpage,times,pgf,amsmath,natbib}
\usepackage{amsthm}
\usepackage{authblk}

\usepackage{setspace}

\doublespacing

\theoremstyle{theorem}
\newtheorem{theorem}{THEOREM}[section]
\theoremstyle{proposition}
\newtheorem{proposition}{PROPOSITION}[section]

\newcommand{\child}[1]{\ensuremath{\mathrm{child}(#1)}}
\newcommand{\tsum}{{\textstyle\sum}}
\newcommand{\tprod}{{\textstyle\prod}}

\title{Modeling mammalian methylome evolution}
\date{}

\author[1]{Jianghan Qu }
\author[1]{Andrew D. Smith}
\affil[1]{Molecular and Computational Biology Section, Division
          of Biological Sciences, University of Southern California,
          Los Angeles, CA 90089, USA}



\begin{document}

\maketitle

\section{State space and units of measurement for DNA methylation}

%%% Talk about state space.
DNA methylation is often discussed in terms of levels at individual
CpG sites, and the level reflects the fraction of cells that have the
discrete methyl mark at that site (more accurately molecules with the
mark, in the case of non-haploid cells). In multiple studies since
2009, when considering cells that are relatively pure in terms of
phenotype, the methylation levels have been observed to fall into two
categories: high and low levels. This is seen in the global bi-modal
distributions of methylation levels, and when one observes profiles of
DNA methylation in a genome browser. There are special cases of
intermediate methylation levels, for example cancers have large
domains of partial methylation. In addition, imprinted loci have
intermediate methylation (one allele methylated through an imprinting
control region in most somatic cells). However, for the vast majority
of the sites, major phenotypic differences among healthy cells usually
involve methylation changing from low to high, or from high to
low. For this reason, all our modeling is in terms of low and high
methylation states, and we use the corresponding state space $\{0,
1\}$. This allows for a distribution of the observed levels associated
with the ``low'' methylation state, and another distribution for the
observed levels at sites occupying the ``high'' state.

Although this restriction is justified, our modeling approach still
allows for an interpretation that is consistent with the fact that
methylation is usually measured as levels between 0 and 1. One may
consider that level as (an unbiased estimator for) the probability
that a randomly sampled molecule would have a methyl group at the
site. In our modeling we ultimately make use of probabilities over the
state space, which behave very much like a continuous level. At the
same time, the discrete state space is highly convenient, and the
preponderance of evidence indicates that in most cases the state
carries almost the same information as the level (and may be more
robust to artifacts and random noise).

Current technology for measuring DNA methylation can interrogate
single nucleotides (in mammals, with specific exceptions, these are
cytosines of CpG dinucleotides). In modeling how methylation status
relates between species we must use measurements that can be
considered orthologous. Part of our analyses uses 200 base bins,
assigning a methylation state to the bin, and using the average of
individual site methylation levels to obtain a level for the bin, when
necessary. We also devise a strategy to model individual CpG sites
directly, even when those sites are not mutually orthologous within
all considered species. Below we explain our modeling approach in
terms of CpG sites, but the same approaches can be applied for any
kind of unit to which a methylation level may be assigned.

\section{Model identifiability}

Markov models on evolutionary trees have been proved useful and widely
applied in phylogenetic studies. An evolutionary Markov model consists
of a tree topology and a set of transition probability matrices on the
branches of the tree. Identifiability of the tree topology from the
distribution of character states at terminal nodes have been
established
\citep{chang1992reconstruction,steel1995reconstructing}.
Under some mild conditions, the full model is identifiable from
knowledge of the joint distribution of character states at triples of
terminal nodes of the tree \citep{chang1996full}. It is a common
practice to model the evolution process with a time-reversible Markov
process, which makes the phylogenetic model satisfy conditions for
full model identifiability \citep{lio1998models}. A recent application
of phylogenetic model to DNA methylation changes in the context of
cell lineage differentiation also assumed reversibility
\citep{capra2014modeling}. However, preliminary results from our
comparative analysis of sperm methylomes across species have shown
that the methylome evolution might be directional and have not reached
equilibrium during recent mammalian evolution. Therefore, we did not
require the Markov process modeling the epigenome evolution to be
reversible. Instead, in order to satisfy model identifiability
conditions, we assumed that one of the two branches at the root is
effectively $0$. Based on previous phylogenetic and archaeological
dating of species divergence time, we set the shorter branch, between
the last common ancestor (LCA) of human and dog and the LCA of human
and mouse, to $0$ to have minimal deviation from the theoretical
binary tree. This is equivalent to assuming a rooted semi-binary tree,
where the root has 3 children and other internal nodes have 2
children.


\section{Phylo-epigenetic model with independent sites}
\label{sec:indep}

%%% Input data: how to get hypoprobs
We begin by assuming that methylation levels at CpG sites are
independent and identically distributed, following a mixture of two
beta distributions. The two components of the mixture distribution
correspond to the low and high methylation
states. Given the state of an individual site, the WGBS read
proportion follows a beta-binomial
distribution. We obtain the maximum likelihood estimates (MLE) of the
mixing proportion and beta distribution parameters using an
Expectation Maximization (EM) algorithm. The M-step of the EM
algorithm is as previously described in a hidden Markov model setting
for identifying hypomethylated regions \citep{molaro2011sperm}. Then
we calculate posterior probabilities of individual CpG sites being in
the ``low'' state. These hypomethylation probabilities serve as input
data for the phyloepigenetic model assuming site-independence.

%%% the Markov process
Our goal is to model the evolution of DNA methylation states in
multiple extant species with a common ancestor. We first focus on a
single-site to derive the likelihood function, and then extend the
model to multiple sites but for which epigenomic evolution is assumed
to be independent.

We assume that the methylation state at a single CpG site evolves
according to a two-state continuous-time Markov process. Let
$\pi=(\pi_0, \pi_1)$ be the initial distribution of the methylation
state at the root node, and let the transition rate matrix be
\[
Q=\begin{bmatrix}
-\lambda & \lambda\\
    \eta & -\eta
\end{bmatrix}.
\]
The transition probability matrix between two time points separated by
time $t$ is $P(t) = \exp(Qt)$, which is determined by two terms
$(t(\lambda + \eta), \lambda/\eta)$. Let $\lambda+\eta = 1$, so that
the mutation rate and branch length parameters are identifiable.

Let $\tau=\{V, E\}$ be the phylogenetic tree with known topology and
unknown branch lengths, where $V$ is the set of known vertices and $E$
is the set of branches with unknown lengths. The model parameter space
is thus
\[
\Theta=\{E,\lambda, \pi\}.
\]

%%% Some notation first
We require some notation for representing nodes and their
relationships in the tree.  We use $r$ to denote the root node of the
phylogenetic tree, and use $u$, $v$ and $w$ to denote 3 consecutive
nodes on a lineage, such that $(u, v)\in E$ and $(v, w)\in E$.  Let
random variable $s(u) \in \{0, 1\}$ be the state of node $u$. Let
$\ell_{v}$ be the length of edge $(u, v)\in E$. Let $X(u)$ be the set
of methylation states at leaf nodes that are descendents of $u$. Thus,
the observed data is represented by $X(r)$.  We use $j$ and $k$ to
denote methylation states. In general, we will use $j$ to denote the
state at the parent of a node whose state is denoted by $k$.


%%% the tree and our notation
For node $v$, whose parent $u$ has methylation state $s(u) = j$, the
probability of observing states $X(v)$ at terminal descendents of node
$v$ is
\[
p_{j}(v) = \Pr(X(v) | s(u) = j, \Theta).
\]
For notational convenience we define
\[
q_k(v) = \left\{
\begin{array}{ll}
  \Pr(s(v) = k) & \mbox{if $v$ is a leaf node,} \\
  \prod_{w\in \child{v}}  p_{k}(w) & \mbox{otherwise,} \\
\end{array}\right.
\]
where $\Pr(s(v) = k)$ is 0 or 1 when methylation state is observed,
and between $(0,1)$ when the observed data is a continuous level
representing a probability distribution of the state space.


We can then write the probability $p_j(v)$ as the recurrence
\begin{equation}\label{eqn:pruning}
p_{j}(v) = \sum_{k}\Big[P(\ell_{v})_{jk} \times q_k(v)\Big].
\end{equation}
The likelihood of the observed data for a single CpG site is thus
%%
\begin{equation}
 L(\Theta;X) = \Pr(X(r)|\Theta) = \sum_{j\in\{0,1\}}\pi_{j}q_j(r).
\end{equation}
The recurrence in \eqref{eqn:pruning} is the basis of Felsenstein's
pruning algorithm for efficiently computing the likelihood of a tree
topology, branch lengths and transition rate, given data at leaf nodes
\citep{felsenstein1981evolutionary}.

Moving from single site to multiple sites, let $N$ be the total number of
sites in the methylome.  Let $X_n (1\leq n\leq N)$ be the set of
methylation states associated with leaf nodes at site
$n$. $X=X_1\ldots X_N$ denote the observed methylation states at all
leaf nodes. Since methylation states at distinct sites of the
methylome evolve independently, the likelihood for observed data at
multiple sites is
\[
L(\Theta|X) = \prod_{n=1}^NL(\Theta|X_n).
\]

In our implementation, we optimize parameters using gradient descent.
The likelihood and gradients are recursively computed in the same
spirit of the pruning algorithm
\citep{felsenstein1981evolutionary}. With the MLE of model parameters,
we then compute the maximum likelihood reconstruction of the HME at
each CpG site. The reconstruction can be achieved through a dynamic
programming algorithm with time complexity linear to the number of
nodes in the tree \citep{pupko2000fast}.

\section{Phylo-epigenetic model with dependent sites}

We develop a model to allow for two processes that jointly
describe the observed mammalian methylome: one is methylation state
inheritance from ancestral species, and the other is the correlation
observed between neighboring sites within a species.

The inheritance process is defined by $Q$, or equivalently $P(t)$, as
previously introduced. The correlation between neighboring CpG sites
in a species is described with a discrete-time (corresponding to CpG
sites) Markov chain over the state space $\{0,1\}$. Let the initial
distribution be $\pi$ in the root species. The transition probability
matrix,
\[
G=\begin{bmatrix}
      g_0  & 1-g_0  \\[0.3em]
      1-g_1 & g_1
    \end{bmatrix}
\]
is assumed to be homogeneous in all species (this assumption can be
relaxed).

As in Section~\ref{sec:indep}, we use $\tau=\{V, E\}$ to denote the
phylogenetic tree relating species, with known topology and unknown
branch lengths. The model parameter space is thus
\[
\Theta=\{\tau, Q, G, \pi_0\}.
\]

The two Markov processes are combined in the following way to model
the evolution of $N$ consecutive CpG sites in a genomic region. The
CpG sites are ordered from $1$ to $N$ by their occurrences from the 5'
end of the `+' strand of a chosen primary reference genome (human
reference genome in this study).

Consider two neighboring CpG sites $n-1$ and $n$, and a pair of nodes $(u, v)$ linked by a branch in the
phylogenetic tree, where $u$ is the ancestor of $v$. Let random variable
$v_n$ be the methylation state of site $n$ in species
$v$. The conditional distribution of $v_j$ given the methylation
state of previous CpG site $v_{n-1}$ and the ancestral methylation state
$u_n$ is defined as:
\begin{equation}\label{eqn:unittrans}
p_v(i,j,k) = \Pr(v_n=k|v_{n-1}=i,u_n=j)=  \frac{G_{ik}P(\ell_v)_{jk}}{\sum_{k'=0,1}G_{ik}P(\ell_v)_{jk'}},~\forall i,j,k\in\{0,1\}.
\end{equation}

Special cases involve sites at position $n=1$, and sites in the root
node.  The root methylome is modeled only with the discrete-time
Markov chain for site-interdependence, $M_{dep}\{\pi, G\}$ .  For
sites at position $n=0$, their methylation state evolution is modeled
only with the continuous-time Markov process for inheritance and
mutation, $M_{inh}\{\pi, Q, \tau\}$.

\paragraph{Complete data likelihood}
We examine the the complete data likelihood, from which
we later derive approximation method for model learning. Assume the
methylation states at all sites in each external and internal species
are observed, denoted by $O$. Given model parameters
$\Theta=\{\tau, Q, G, \pi_0\}$, the complete data likelihood
is
\[
L(\Theta; O) =
\Pr(O|\Theta) = \Pr(O_1|\Theta) \prod_{n=2}^{N}G_{o_{n-1}(r)o_{n}(r)}\prod_{u\in \textit{I}} \prod_{v\in \child{u}} p_v(O_{n-1}(v), O_{n}(u), O_{n}(v))
\]
where \textit{I} is the set of internal nodes, and $O_{n}(v)$ is the methylation state of node $v$ at site $n$.

\begin{equation}\label{eq:complik}
\begin{aligned}
 \log L(\Theta; O) =
 & \log\pi_{O_1(r)} + \sum_{v\neq r}\sum_{j,k\in\{0,1\}}w_{jk}(v)P(\ell_v)_{jk} + \\
 & \sum_{i,k\in\{0,1\}}w_{ik}\log(G_{ij}) + \sum_{v\neq r}\sum_{i,j,k\in\{0,1\}}w_{ijk}(v) p_v(i,j,k)
\end{aligned}
\end{equation}
where
\begin{equation}\label{eq:sufstat}
\begin{aligned}
w_{ijk}(v) &= \sum_{n=2}^N I\{O_{n-1}(v)= i, O_{n}(u)=j, O_{n}(v)=k\}\\
w_{jk}(v) &= I\{O_{n}(u)=j, O_{n}(v)=k\} \\
w_{ik} &= \sum_{n=2}^N I\{O_{n-1}(r)= i, O_{n}(r)=k\}
\end{aligned}
\end{equation}
where $I$ is an indicator function. Therefore, $W=\{ w_{jk}(v),
w_{ijk}(v), w_{ik} : i, j, k\in\{0,1\}, v\in
\textit{V}\backslash\{r\}\}$ are the sufficient statistics for the
model parameters. The MLE for model parameters can be efficiently
obtained through numerical methods, such as gradient descent, given
complete data.

\paragraph{Hidden Markov model representation}
The ancestral methylomes are unobserved, which we hope to infer using
observed data from the extant species and their phylogenetic
relationship. We can interpret our interdependent-site phyloepigenetic
model as a hidden Markov model over the state space
\[
H=\{h\in\{0,1\}^{|V|}: \text{all possible combination of
methylation states over the phylogenetic tree} \}.
\]
Let the $H_n$ be a random variable denoting the HME at site $n$.  Each
state specifies the history of the methylation evolution a single
genomic site, and thus we call it ``History of Methylation Evolution''
(HME). The transition probability between two HMEs can be calculated
using quantities defined in the previous section. Let $h, h' \in H$ be
the HMEs at two neighboring sites in the genome. Let $u(h)$ be the
methylation state at node $u$ as specified by HME $h$. Let $v$ be an
internal or leaf node in the phylogenetic tree, and let its parent
node be $u$. The transition probability from HME $h$ to $h'$ is
\begin{equation}
\Pr(H_{n+1} = h'|H_n=h) = G_{r(h)r(h')}\prod_{(u,v)\in E}p_v(v(h),u(h'),v(h')),
\end{equation}\label{eqn:deftrans}
where $p_v(v(h),u(h'),v(h'))$ is defined in \eqref{eqn:unittrans}. Use
$A=\{a_{hh'}: h,h'\in H \}$ to denote the transition probability
matrix between HME states.

The data likelihood calculation can be written as
\begin{equation}
L(\Theta; X) = \sum_{H_1,\ldots, H_N\in H} \Pr(H_1)\Pr(X_1|H_1)\prod_{n=1}^{N-1} \Pr(H_{n+1}| H_{n})\Pr(X_{n+1}|H_{n+1}).
\end{equation}\label{L}

\paragraph{Bayesian network interpretation}
The complete data likelihood in \eqref{eq:complik} is factorized by
conditional probabilities, as our model naturally defines a dynamic
Bayesian network (DBN). The history of evolution (HME) of each site in
the methylome corresponds to a time-slice in DBN. Each site in each
species corresponds to a node, a random variable for methylation
state, in the graph. The interdependence between neighboring sites and
inheritance relationship between ancestor and descendant sites define
directed edges between nodes. With observations made at a subset of
the nodes in the network (nodes associated with leaf species), we aim
to learn the model parameters.  Given the sufficient statistics in
\eqref{eq:sufstat}, the MLE of model parameters can be derived by
maximizing \eqref{eq:complik}. Given the model parameters
$\theta$, and observations from the leaf nodes $X$, theoretically we
can derive the joint distribution of the hidden variables (internal
node methylation states $Z$) conditional on the observed leaf data. We
refer to the probability distributions conditional on the observed
data as posterior distributions. Using the joint posterior
distribution we can obtain expected values of the sufficient
statistics $E_{Z|X,\theta}W$. These two processes are exactly the two
steps in Expectation-Maximization (EM) algorithm.

\paragraph{E-step:}
\begin{equation}
  \begin{aligned}
    Q(\theta|\theta^{(t)}) = & E_{Z|X,\theta^{(t)}} \log(L(\theta; X, Z )) \\
    = & \sum_{v\neq r}\sum_{j,k\in\{0,1\}}E(w_{jk}(v)|X, \theta^{(t)}) P(\ell_v)_{jk} + \\
    & \sum_{i,k\in\{0,1\}}E(w_{ik}|X, \theta^{(t)})\log(G_{ik}) + \\
    & \sum_{v\neq r}\sum_{i,j,k\in\{0,1\}}E(w_{ijk}(v)|X, \theta^{(t)}) p_v(i,j,k)
  \end{aligned}
\end{equation}
\paragraph{M-step:}
\[
\theta^{(t+1)} = \underset{\theta}{\mathrm{argmin}} Q(\theta|\theta^{(t)})
\]
where $L(\theta; X, Z)$ is the complete data likelihood defined in
 \eqref{eq:complik}.
M-step can be effectively solved by gradient-descent or other
numerical methods.

\paragraph{Metropolis-Hasting algorithm}
For E-step, we use Markov Chain Monte Carlo (MCMC) method,
Metropolis-Hasting algorithm in particular, to obtain the expectation
of the sufficient statistics given model parameters and observed data
at leaf nodes.

\begin{itemize}
\item Start from a specific initiation of the states at all nodes in
internal species, denoted with $Z_0$.
\item Iterate through all nodes in the graph, from site 1 to site $N$,
and for each site according to a post-order traversal of species in
the phylogenetic tree. At each node $z$ in the graph, we make a
proposal to flip its state, i.e. $z^{prop}=1-z^{(t)}$.  We accept the
proposal $z^{(t)} = z^{prop}$ with probability
\[
\alpha=min\{1, \frac{P(z^{prop}| MB(z,t-1))}{P(z^{(t-1)}|MB(z,t-1))}\},
\]
where $MB(v)$ denotes the Markov blanket of node $v$ in DBN, and $MB(z,t)$ denotes the value of these variables in the $t^{\text{th}}$ sample in MCMC.
If the proposal is rejected, let $z^{(t)}= z^{(t-1)}$. After we
iterate through all the hidden nodes in the graph, we have generated
an new sample $Z_{t}$.
\item From sample $\{X, Z_{t}\}$, we can obtain $W_{t}$ as an
estimator of $E_{Z|X,\theta}W$.
\end{itemize}

\paragraph{Measure chain convergence.}
The MCMC is guaranteed to converge to the target distribution $\Pr(Z|
X, \theta)$. To determine convergence time, we let two chains from
different starting points run independently and measure the difference
between the two chains each time new samples are generated. Our goal
is to approximate $E_{Z|X,\theta}W$. Let $W'_{t}$ and $W''_{t}$ be the
estimators from the two chains. Notice that
\[
\sum_{ijk} w_{ijk}(v) = N,~~\forall v\in\textit{V}\backslash\{r\}),
\]
therefore $\frac{1}{N}\{w_{ijk}(v)\}$ is a probability vector, which
for simplicity we also denote as $w_{ijk}(v)$. To measure the
difference between $W'_{t}$ and $W''_{t}$, we use Kullback-Leibler(KL)
divergence between the probability vectors associated with each
internal node $v$:
\[
KL_t(v) = D_{KL}(w'_{ijk}(v) || w''_{ijk}(v)), ~~ v\in\textit{V}\backslash\{r\}.
\]
For a small $\epsilon\in(0,1)$, we consider the two chains to be mixed
at time
\[
t_{mix}= \underset{t}{\mathrm{argmin}} \{t: \max\{KL_t(v): v\in\textit{V}\backslash\{r\}\} < \epsilon\}.
\]
We stop the chains at time $t_{mix}$, and use $W'_{t_{mix}}$ as an
approximation for $E_{Z|X,\theta}W$. In our implementation, $\epsilon=1e-4$.



\paragraph{Computing approximate posterior probability for individual sites}
We use the Markov blanket of each node to update its marginal
posterior.  Let $v_{n}$ be the methylation state at the $n^{th}$
position in species $v$ in the phylogenetic tree. Let $B(v_{n})$ be
the set of joint states for nodes in $v_{n}$'s Markov blanket. The
joint distribution of nodes in the Markov blanket is approximated with
the product of their marginal distributions.  The approximated
probability distribution is denoted as $p_{b}(v_n), b\in B(v_{n})$
below.

There are 9 types of nodes in the network, for each of which we
describe the composition of the Markov blanket, and the approximation
procedure.

\begin{enumerate}
\item[Case 1:] Root species $r$, site $n=1$. Its Markov blanket
includes three children $l_{n},m_{n} r_{n}$, and the neighboring site
$r_{n+1}$. For each state of the Markov blanket, the normalized
conditional hypomethylation probability is calculated by
\begin{equation*}
  \begin{aligned}
   \Pr(r_{n} = 0|b) \propto & \prod_{c \in \child{r}}\Pr(c_{n}| r_{n} = 0)
  \end{aligned}
\end{equation*}
And the probability distribution of $r_n$'s Markov blanket $B(r_{n})$ is
\[
p_{b}(r_{n}) = \prod_{c \in \child{r}}p_{c_{n}}\times p_{r_{n+1}}, \forall~b\in B(v_{n}).
\]

\item[Case 2:] Root species $r$, site $n=N$. The
Markov blanket includes $\{c_{m}: c\in \child{r}, m=n-1, n.\}$ and $r_{n-1}$.
\begin{gather*}
p_b(r_n) = \prod_{c \in \child{r}}p_{c_{n}} p_{c_{n-1}} \\
\Pr(r_{n} = 0|b) \propto \Pr(r_{n} = 0| r_{n-1})\prod_{c \in \child{r}}p_c(c_{n-1}, 0, c_{n})
\end{gather*}
\item[Case 3:] Root species $r$, site $1< n < N$. The Markov
blanket includes children states $\{c_{n},c_{n-1}: c\in
\child{r}\}$, and neighboring states $\{r_{n-1}, r_{n+1}\}$.
\begin{gather*}
p_b(r_n) = \prod_{c\in \child{r}} p_{c_{n}}p_{c_{n-1}}\times
p_{r_{n-1}}p_{r_{n+1}}. \\
\Pr(r_{n} = 0|b) \propto \Pr(r_{n} = 0| r_{n-1})
    \Pr(r_{n+1}|r_{n} = 0)\prod_{c \in \child{r}}p_c(c_{n-1}, 0, c_n)
\end{gather*}
\item[Case 4:] Species $v\in \mathcal{L}$ is a leaf species, and site
$n=1$.  Its Markov blanket includes $v_{n+1}$, and $u_{n}, u_{n+1}$,
where $u$ is the parent of node $v$.
\begin{gather*}
p_b(v_n) = p_{v_{n+1}}p_{u_{n}}p_{ u_{n+1}}\\
\Pr(v_{n} = 0|b)
  \propto \Pr(v_{n} = 0| u_{n}) p_v(0, u_{n+1}, v_{n+1})
\end{gather*}

\item[Case 5:] Species $v\in \mathcal{L}$ is a leaf species, and $n=N$. Its
Markov blanket includes $v_{n-1}$ and $u_{n}$, where $u$ is the
parent of node $v$.
\begin{gather*}
 p_b(v_n) = p_{v_{n-1}}p_{u_{n}} \\
  \Pr(v_{n} = 0|b)\propto p_v(v_{n-1}, u_{n}, 0)
\end{gather*}
\item[Case 6:] Species $v \in \mathcal{L}$ is a leaf species, and site $1 < n < N$. The Markov blanket includes $v_{n-1}$, $v_{n+1}$, $u_{n}$ and $u_{n+1}$.
  \begin{gather*}
    p_b(v_n) = p_{v_{n-1}}p_{v_{n+1}}p_{u_{n}}p_{u_{n+1}}\\
    \Pr(v_{n} = 0|b) \propto p_v(v_{n-1}, u_{n}, 0)  p_v(0, u_{n+1}, v_{n+1})
  \end{gather*}
\item[Case 7:] Species $v\in V \backslash\{ \{r\} \cup \mathcal{L}\}$
is neither root nor leaf, and site $n=1$. Its Markov blanket includes
$\{c_{n}: c\in \child{v}\}$, $v_{n+1}$, $u_{n}$ and $u_{n+1}$.
\begin{gather*}
  p_b(v_n) = p_{v_{n+1}}p_{u_n}p_{u_{n+1}}\prod_{c\in \child{v}} p_{c_n}\\
  \Pr(v_{n} = 0|b)\propto \Pr(v_{n} = 0| u_{n}) p_v(0, u_{n+1}, v_{n+1})
                          \prod_{c\in \child{v}} \Pr(c_{n} = 0| v_{n}=u)
\end{gather*}
\item[Case 8:] Species $v\in V\backslash\{\{r\} \cup \mathcal{L}\}$ is
neither root nor leaf, and site $n=N$. Its Markov blanket includes
$\{c_{n},c_{n-1}: c\in \child{v}\}$, $v_{n-1}$ and $u_{n}$.
\begin{gather*}
p_b(v_n)= p_{v_{n-1}}p_{u_{n}}\prod_{c\in\child{v}} p_{c_{n-1}}p_{c_{n}} \\
\Pr(v_{n} = 0|b)\propto p_v(v_{n-1}, u_{n}, 0) \prod_{c\in \child{v}} p_c( c_{n-1}, 0, c_{n})
\end{gather*}
\item[Case 9:] Species $v\in V\backslash\{\{r\} \cup  \mathcal{L}\}$ is neither root nor leaf, and site $1< n < N$. The
Markov blanket of $v_n$ includes $v_{n-1}$, $\{c_{n},c_{n-1}: c\in
\child{v}\}$, $v_{n+1}$, $u_{n}$ and $u_{n+1}$.
  \begin{gather*}
 p_b(v_n)= p_{v_{n-1}}\prod_{c\in\child{v}}p_{c_{n-1}}p_{c_n}p_{v_{n+1}}p_{u_{n}}p_{u_{n+1}}\\
 \Pr(v_{n} = 0|b) \propto p_v(v_{n-1}, u_{n}, 0) p_v(0, u_{n+1}, v_{n+1}) \prod_{c\in \child{v}} p_c(c_{n-1}, 0, c_n)
  \end{gather*}
\end{enumerate}

The posterior hypomethylation probability of site $n$ of species $v$
can be approximated with
\[
p'_{v_{n}} = \sum_{b\in B(v_n)}p_{b}\Pr(v_{n} = 0| b).
\]

\paragraph{Summary of model inference procedure}
Together, the procedure for model parameter estimation and
ancestral state reconstruction is summarized as below:
\begin{enumerate}
\item Choose start point for model parameters $\theta^{(t)}$.
\item Iterate the following EM procedure
  \begin{itemize}
  \item E-step: Use Metropolis-Hasting algorithm to approximate
    $E_{Z|X,\theta^{(t)}}W(X,Z)$.
  \item M-step: update model parameters to $\theta^{(t+1)}$.
  \end{itemize}
  until convergence: $||\theta^{(t+1)}-\theta^{(t)}|| <
  \epsilon$.
\item Estimate posterior hypomethylation probability at internal nodes
  using Markov blanket. Use MAP state as the inference for ancestral
  methylation states.
\end{enumerate}


\newpage

\bibliographystyle{agsm}
\bibliography{biblio}

\end{document}
